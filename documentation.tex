\documentclass{article}
\usepackage[utf8]{inputenc}
\usepackage{hyperref}
\usepackage{listings}
\usepackage{color}

\title{TPOT AutoML Flask Application Documentation}
\author{}
\date{}

\begin{document}

\maketitle

\section{Introduction}
This document provides detailed documentation for the TPOT AutoML Flask Application. This Flask application offers a web interface for automatically training machine learning models using the TPOT library on user-uploaded CSV datasets.

\section{Features}
\begin{itemize}
    \item \textbf{File Upload:} Users can upload CSV files to be used as datasets for model training.
    \item \textbf{Feature Specification:} Users can specify which columns to use as features and target in the dataset.
    \item \textbf{Model Training:} The application trains a model using TPOT, an automated machine learning tool.
    \item \textbf{Model Evaluation:} It displays the accuracy of the trained model.
    \item \textbf{Model Download:} Users can download the trained model for future use.
\end{itemize}

\section{Requirements}
The application requires the following Python libraries:
\begin{itemize}
    \item Flask
    \item Pandas
    \item NumPy
    \item scikit-learn
    \item TPOT
    \item Werkzeug
\end{itemize}

\section{Setup and Installation}
\subsection{Environment Setup}
It is recommended to use a virtual environment. Create and activate it as follows:
\begin{lstlisting}[language=bash]
  python -m venv venv
  source venv/bin/activate
\end{lstlisting}

\subsection{Install Dependencies}
Install the required Python libraries using pip:
\begin{lstlisting}[language=bash]
  pip install flask pandas numpy scikit-learn tpot werkzeug
\end{lstlisting}

\subsection{Running the Application}
Run the application using Flask:
\begin{lstlisting}[language=bash]
  export FLASK_APP=app.py
  flask run
\end{lstlisting}
Access the web interface at \url{http://127.0.0.1:5000/}.

\section{Application Structure}
\subsection{app.py}
\begin{itemize}
    \item \textbf{Initialization:} Sets up the Flask application and configuration.
    \item \textbf{Routes:}
    \begin{itemize}
        \item \texttt{/}: Main page for file upload.
        \item \texttt{/upload}: Handles the uploading of CSV files.
        \item \texttt{/train}: Processes the uploaded file and trains a model using TPOT.
        \item \texttt{/results}: Displays the results, including model accuracy.
        \item \texttt{/download\_model}: Allows downloading of the trained model.
    \end{itemize}
\end{itemize}

\subsection{Utility Functions}
\begin{itemize}
    \item \texttt{allowed\_file(filename)}: Checks if the uploaded file is in an allowed format (CSV).
    \item \texttt{preprocess\_data(X\_train, X\_test)}: Preprocesses the training and test data, applying feature scaling and encoding.
\end{itemize}

\section{Security and Limitations}
\begin{itemize}
    \item The application uses Flask's built-in server, which is not suitable for production.
    \item File uploads should be handled carefully to avoid security risks. The application currently only allows CSV files.
    \item The secret key for Flask should be set to a secure, random value in production.
\end{itemize}

\section{Further Development}
\begin{itemize}
    \item Enhance security for file uploads and data handling.
    \item Implement progress tracking for model training.
    \item Consider deploying with a production-ready server like Gunicorn.
\end{itemize}

\end{document}
